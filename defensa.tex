
\documentclass{beamer}

\usepackage[spanish]{babel}
\usepackage[utf8x]{inputenc}
\usepackage{nth}
\usepackage{alltt}
\usepackage{mdframed}
\usepackage{verbatim}
\usepackage{minted}

\title{SISTEMA DE CONTROL Y SEGUIMIENTO DE VEH\'ICULOS DE LA CORPORACI\'ON COMTECO}
\author{Jos\'e Benjam\'in P\'erez Soto}
\institute{Universidad Mayor de San Sim\'on \\ Carrera de Licenciatura en Inform\'atica}
\date{Septiembre, 2015}
\logo{\includegraphics[height=1.5cm]{images/umss.jpg}}

\begin{document}

\frame{\titlepage}

\begin{frame}
\frametitle{Indice}
\tableofcontents
\end{frame}

\begin{frame}
  \frametitle{Antecedentes}
  \section{Antecedentes}
  \nth{1}: Los registros se lo hacen manualmente en papel
  \begin{center}
    \includegraphics[height=3.0in]{images/chapter1/planilla.jpg}
  \end{center}
\end{frame}

\begin{frame}
  \frametitle{\nth{2}: Se transcribe en una hoja de c\'alculo}
  \begin{center}
    \includegraphics[width=\textwidth]{images/chapter1/planilla_excel.png}
  \end{center}
\end{frame}

\begin{frame}
  \frametitle{\nth{3}: A los archivos se los nombra con la fecha de ese d\'ia}
  \begin{center}
    \includegraphics[width=\textwidth]{images/chapter1/carpeta_excel.png}
  \end{center}
\end{frame}

\begin{frame}
  \frametitle{\nth{4}: Estan en carpetas con el nombre del mes}
  \begin{center}
    \includegraphics[width=\textwidth]{images/chapter1/carpeta2.png}
  \end{center}
\end{frame}

\begin{frame}
  \section{Objetivos}
  \frametitle{Objetivos}
  {\bfseries Objetivo General}

  Sistematizar el control de veh\'iculos y de personas para realizar el control
  y gesti\'on de medidas preventivas y correctivas en seguridad de activos y
  valores de la empresa.\\

  {\bfseries Objetivos Especificos}
  \begin{itemize}
      \item Registrar en los parqueo la salida y retorno de los veh\'iculos
      \item Registrar en las oficinas el ingreso de personas
      \item Programar mantenimiento a los veh\'iculos
      \item Enviar notificaciones
  \end{itemize}
\end{frame}

\begin{frame}
  \section{Metodolog\'ia}
  \frametitle{Metodolog\'ia}
  \begin{center}
    {\huge BDD} \\
    Behaviour Driven Development \\
    \vspace{1cm}
    {\it Desarrollo Guiado por Comportamiento}

    {\it ``BDD es una metodología de desarrollo ágil de software que fomenta la colaboración
    entre desarrolladores, testers y clientes.''}
  \end{center}
\end{frame}

\begin{frame}
  \frametitle{Ventajas}
  \begin{itemize}
    \item Fácil de entender
    \item Fácil de leer
    \item Fácil de {\bfseries discutir}
  \end{itemize}
\end{frame}

\begin{frame}[fragile]
  \frametitle{Sintaxis de una característica o feature}
  \begin{verbatim}
    # language: es

  Característica: [nombre de la funcionalidad]
    Como [un actor/rol]
    Para [algún beneficio]
    Quiero [una característica]

    Esquema del escenario: [nombre del escenario]
      Dado [una condición previa]
      Cuando [una acción]
      Entonces [un resultado esperado]

    Esquema del escenario: [...]
      ...
    Esquema del escenario: [...]
      ...
  \end{verbatim}
\end{frame}

\begin{frame}[fragile]
  \frametitle{Ejemplo: Inicio de sesión}
  \begin{minted}{Gherkin}

  # language: es

  Caracteristica: Inicio de sesion en el sistema
      Como un guardia
      Quiero ingresar al sistema
      Para registrar ingresos de personas y vehiculos

      Esquema del escenario: Iniciar sesion
          Dado que ingreso al sistema por el navegador
          Y voy a la opcion "Ingresar"
          Y entro con mi nombre de usuario
          Y mi password
          Cuando oprima el boton "Ingresar"
          Entonces ingreso al sistema
  \end{minted}
\end{frame}

\begin{frame}
  \frametitle{Herramientas}
  \section{Herramientas}
  \begin{itemize}
      \item Behave
      \item Framework Django
      \item Django-Behave
      \item Splinter
  \end{itemize}
\end{frame}

\begin{frame}
  \frametitle{Behave}
  \section{Behave}
  \begin{center}
    {\it Es BDD al estilo de Python}
  \end{center}
  Behave usa los escenarios y los respalda con c\'odigo escrito en Python.
\end{frame}

\begin{frame}
  \frametitle{}
  \begin{center}
    Veamos la ejecuci\'on usando:\\
    \vspace{1cm}
    {\huge Behave}
  \end{center}
\end{frame}

\begin{frame}
  \frametitle{Framework Django}
  Django es un entorno de desarrollo web escrito en Python que fomenta el
  desarrollo rápido y el diseño limpio.
  \begin{columns}

    \column{0.5\textwidth}
      \includegraphics[height=3.0in]{images/chapter5/Django_mvc.png}

    \column{0.5\textwidth}
      \includegraphics[height=3.0in]{images/chapter5/appsdjango2.png}
  \end{columns}
\end{frame}

\begin{frame}
  \frametitle{Pruebas Finales}
  \section{Pruebas Finales}
  Con BDD se vio:
  \begin{itemize}
      \item Escenarios
      \item Pruebas antes del desarrollo del sistema
  \end{itemize}
\end{frame}

\begin{frame}
  \frametitle{Herramientas}
  \begin{itemize}
      \item Django-Behave
      \item Splinter
  \end{itemize}
\end{frame}

\begin{frame}[fragile]
  \frametitle{Configuraci\'on de las aplicaciones}
  \begin{verbatim}
flosite
    |-- manage.py
    |-- flosite (carpeta principal)
    |   `-- ...
    |-- accounts/
    |  |-- features
    |  |   |-- behave.ini
    |  |   |-- environment.py
    |  |   |-- login.feature
    |  |   `-- steps
    |  |       `-- step_login.py
    |  |-- __init__.py
    |  |-- models.py
    |  |-- urls.py
    |  `-- views.py
    |-- ...

  \end{verbatim}
\end{frame}

\begin{frame}
  \frametitle{}
  \begin{center}
    Veamos una demostraci\'on
  \end{center}
\end{frame}

\begin{frame}
  \frametitle{Conclusiones}
  \section{Conclusiones}
  \begin{itemize}
      \item El uso de BDD fue excelente al momento de entender las funcionalidades.
      \item El uso de herramientas libres ayudo a no tener problemas de licencias.
  \end{itemize}
\end{frame}

\begin{frame}
  \frametitle{Recomendaciones}
  \section{Recomendaciones}
  \begin{itemize}
      \item Es importante tener informaci\'on t\'ecnica de los servidores.
  \end{itemize}
\end{frame}

\begin{frame}
  \frametitle{}
  \section{Preguntas}
  \begin{center}
    Preguntas
  \end{center}
\end{frame}

\begin{frame}
  \frametitle{}
  \begin{center}
    {\huge GRACIAS}
  \end{center}
\end{frame}

\end{document}

