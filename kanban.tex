\documentclass[xcolor=dvipsnames]{beamer}
\usecolortheme[RGB={47,96,165}]{structure}
%\usepackage[activeacute,spanish]{babel}
%\usepackage[spanish]{babel}
\usepackage[utf8x]{inputenc}

\setbeamertemplate{blocks}[rounded][shadow=true]

%\usetheme{Singapore}  %Sombreado, denso * ; Warsaw ; Singapore
\usetheme{Warsaw}
\setbeamercovered{transparent}
\usefonttheme{structuresmallcapsserif}


\title{Kanban}
\author{Benjam\'in P\'erez}
\institute[SCESI]{Sociedad Cient\'ifica de Estudiantes de Sistemas Inform\'atica \\ $\ $ \\ \includegraphics[height=0.7cm,hiresbb=true]{images/banner}}
\date[\today]{Happy Hacking ;-)}
\logo{\includegraphics[height=1cm,hiresbb=true]{images/scesi02_01.png}}

\begin{document}

\frame{\titlepage}

\begin{frame}{Que es?}
  \begin{center}
    Kanban, es una t\'ecnica para la gestion de un proceso de desarrollo de softwre.
  \end{center}
\end{frame}

\begin{frame}{Proceso de desarrollo}
  Un proceso de desarrollo de software puede ser pensado como una tuber\'ia.
  \begin{center}
      \includegraphics[height=2cm,hiresbb=true]{kanban/software-pipeline.png}
  \end{center}
  \begin{itemize}
      \item Analisis
      \item Desarrollo
      \item Test
  \end{itemize}
\end{frame}

\begin{frame}{El efecto del cuello de botella}
  En una tuber\'ia el cuello de botella restringe el flujo y el rendimiento en conjunto es limitado.
  \begin{center}
      \includegraphics[height=2cm,hiresbb=true]{kanban/bottleneck-inventory.png}
  \end{center}
\end{frame}

\begin{frame}{Tablero de trabajo}
  \begin{center}
      \includegraphics[height=4cm,hiresbb=true]{kanban/kanban-board-0.png}
  \end{center}
  Los n\'umeros que se encuentran en la parte superior de cada columna es el l\'imite de tareas.
\end{frame}

\begin{frame}{Ejemplo de trabajo}
  \begin{center}
      \includegraphics[height=4cm,hiresbb=true]{kanban/kanban-board-1.png}
  \end{center}
  Donde esta el problema?
\end{frame}

\begin{frame}{Ejemplo de trabajo}
  \begin{center}
      \includegraphics[height=4cm,hiresbb=true]{kanban/kanban-board-2.png}
  \end{center}
\end{frame}

\begin{frame}{Ejemplo de trabajo}
  \begin{center}
      \includegraphics[height=4cm,hiresbb=true]{kanban/kanban-board-3.png}
  \end{center}
\end{frame}

\begin{frame}{Herramientas}
  \framesubtitle{taiga.io}
  \begin{center}
      \includegraphics[height=4cm,hiresbb=true]{kanban/taiga.png}
  \end{center}
\end{frame}

\begin{frame}{Taiga y Kanban}
  \begin{center}
      \includegraphics[height=5cm,hiresbb=true]{kanban/kanban-screen.png}
  \end{center}
\end{frame}

\begin{frame}{http://taiga.io}
  Veamos una demostraci\'on.
\end{frame}

\begin{frame}{Fin}
  Gracias.
\end{frame}

\end{document}
