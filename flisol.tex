
\documentclass{beamer}

\usepackage[spanish]{babel}
\usepackage[utf8x]{inputenc}
\usepackage{nth}
\usepackage{alltt}
\usepackage{mdframed}
\usepackage{verbatim}
\usepackage{minted}

\title{Django con BDD}
\author{Benjam\'in P\'erez}
\institute{Universidad Mayor de San Sim\'on \\ Festival Latinoamericano de Instalación de Software Libre \\ $\ $ \\ \includegraphics[height=0.7cm,hiresbb=true]{images/banner}}
\date{Abril, 2016}
\logo{\includegraphics[height=1.5cm]{images/scesi02.png}}

\begin{document}

\frame{\titlepage}

\begin{frame}
  \frametitle{Como se hace la entrega del Software}
  \begin{center}
    \includegraphics[height=3.0in]{images/bdd/img_time_line.png}
  \end{center}
\end{frame}

\begin{frame}
  \frametitle{Como se hace la entrega del Software}
  \begin{center}
    \includegraphics[height=3.0in]{images/bdd/img_planning.png}
  \end{center}
\end{frame}

\begin{frame}
  \frametitle{Como se hace la entrega del Software}
  \begin{center}
    \includegraphics[height=3.0in]{images/bdd/img_analysis.png}
  \end{center}
\end{frame}

\begin{frame}
  \frametitle{Como se hace la entrega del Software}
  \begin{center}
    \includegraphics[height=3.0in]{images/bdd/img_design.png}
  \end{center}
\end{frame}

\begin{frame}
  \frametitle{Como se hace la entrega del Software}
  \begin{center}
    \includegraphics[height=3.0in]{images/bdd/img_code.png}
  \end{center}
\end{frame}

\begin{frame}
  \frametitle{Como se hace la entrega del Software}
  \begin{center}
    \includegraphics[height=3.0in]{images/bdd/img_test.png}
  \end{center}
\end{frame}

\begin{frame}
  \frametitle{Como se hace la entrega del Software}
  \begin{center}
    \includegraphics[height=3.0in]{images/bdd/img_deploy.png}
  \end{center}
\end{frame}

\begin{frame}
  \frametitle{Y cu\'al es el problema?}
  \begin{center}
    \includegraphics[height=2.5in]{images/bdd/problem.png}
  \end{center}
\end{frame}

\begin{frame}
  \frametitle{}
  \begin{center}
    \includegraphics[height=2.5in]{images/bdd/solucion.png}
  \end{center}
\end{frame}

\begin{frame}
  \frametitle{BDD}
  \begin{center}
    {\huge BDD} \\
    Behaviour Driven Development \\
    \vspace{1cm}
    {\it Desarrollo Guiado por Comportamiento}

    {\it ``BDD es una metodología de desarrollo ágil de software que fomenta la colaboración
    entre desarrolladores, testers y clientes.''}
  \end{center}
\end{frame}

\begin{frame}
\frametitle{Seg\'un Dan North}
  \includegraphics[height=0.8in]{images/bdd/dannorth.jpg}
  \begin{center}
    \vspace{0.5cm}
    {\it ``Behaviour-driven development is about implementing an application by describing
    its behaviour from the perspective of its stakeholders''}

    \vspace{0.5cm}

    {\it ``BDD es la implementación de una aplicación mediante la descripción de su comportamiento
      desde la perspectiva de los interesados''}
  \end{center}
\end{frame}

\begin{frame}
  \frametitle{Ventajas}
  \begin{itemize}
    \item Fácil de entender
    \item Fácil de leer
    \item Fácil de {\bfseries discutir}
  \end{itemize}
\end{frame}

\begin{frame}[fragile]
  \frametitle{Sintaxis de una característica o feature}
  \begin{alltt}
\emph{Feature:} [nombre de la característica en general a probar]
  \emph{As} [un actor/rol]
  \emph{In order to} [algún beneficio]
  \emph{I want} [una característica]

  \emph{Scenario:} [primera funcionalidad]
    \emph{Given} [una condición previa]
    \emph{When} [una acción]
    \emph{Then} [un resultado esperado]
\end{alltt}
\end{frame}

\begin{frame}[fragile]
  \frametitle{Sintaxis de una característica o feature}
  \begin{verbatim}
    # language: es

  Característica: [nombre de la funcionalidad]
    Como [un actor/rol]
    Para [algún beneficio]
    Quiero [una característica]

    Esquema del escenario: [nombre del escenario]
      Dado [una condición previa]
      Cuando [una acción]
      Entonces [un resultado esperado]

    Esquema del escenario: [...]
      ...
    Esquema del escenario: [...]
      ...
  \end{verbatim}
\end{frame}

\begin{frame}[fragile]
  \frametitle{Ejemplo: Inicio de sesión}
  \begin{minted}{Gherkin}

  # language: es

  Caracteristica: Inicio de sesion en el sistema
      Como un usuario
      Quiero ingresar al sistema
      Para probar mi cuenta

      Esquema del escenario: Iniciar sesion
          Dado que ingreso al sistema por el navegador
          Y voy a la opcion "Ingresar"
          Y entro con mi nombre de usuario josben
          Y mi password asdf
          Cuando oprima el boton "Ingresar"
          Entonces debo ingreso al sistema
  \end{minted}
\end{frame}

\begin{frame}
  \frametitle{Herramientas}
  \section{Herramientas}
  \begin{itemize}
      \item Behave
      \item Framework Django
      \item Django-Behave
      \item Splinter
  \end{itemize}
\end{frame}

\begin{frame}
  \frametitle{Behave}
  \section{Behave}
  \begin{center}
    {\it Es BDD al estilo de Python}
  \end{center}
  Behave usa los escenarios y los respalda con c\'odigo escrito en Python.
\end{frame}

\begin{frame}
  \frametitle{}
  \begin{center}
    Veamos la ejecuci\'on usando:\\
    \vspace{1cm}
    {\huge Behave}
  \end{center}
\end{frame}

\begin{frame}
  \frametitle{Framework Django}
  Django es un entorno de desarrollo web escrito en Python que fomenta el
  desarrollo rápido y el diseño limpio.
  \begin{columns}

    \column{0.5\textwidth}
      \includegraphics[height=3.0in]{images/chapter5/Django_mvc.png}

    \column{0.5\textwidth}
      \includegraphics[height=3.0in]{images/chapter5/appsdjango2.png}
  \end{columns}
\end{frame}

\begin{frame}
  \frametitle{Pruebas Finales}
  \section{Pruebas Finales}
  Con BDD se vio:
  \begin{itemize}
      \item Escenarios
      \item Pruebas antes del desarrollo del sistema
  \end{itemize}
\end{frame}

\begin{frame}
  \frametitle{Herramientas}
  \begin{itemize}
      \item Django-Behave
      \item Splinter
  \end{itemize}
\end{frame}

\begin{frame}[fragile]
  \frametitle{Configuraci\'on de las aplicaciones}
  \begin{verbatim}
flosite
    |-- manage.py
    |-- flosite (carpeta principal)
    |   `-- ...
    |-- accounts/
    |  |-- features
    |  |   |-- behave.ini
    |  |   |-- environment.py
    |  |   |-- login.feature
    |  |   `-- steps
    |  |       `-- step_login.py
    |  |-- __init__.py
    |  |-- models.py
    |  |-- urls.py
    |  `-- views.py
    |-- ...

  \end{verbatim}
\end{frame}

\begin{frame}
  \frametitle{}
  \begin{center}
    Veamos una demostraci\'on
  \end{center}
\end{frame}

\begin{frame}
  \frametitle{}
  \section{Preguntas}
  \begin{center}
    Preguntas
  \end{center}
\end{frame}

\begin{frame}
  \frametitle{}
  \begin{center}
  \begin{columns}
    \column{0.1\textwidth}
      \includegraphics[height=2cm, width=2cm]{images/github.png}

    \column{0.8\textwidth}
      \url {http://github.com/josben}
  \end{columns}

    {\huge GRACIAS}
  \end{center}
\end{frame}

\end{document}

