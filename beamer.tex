\documentclass[xcolor=dvipsnames]{beamer}
\usecolortheme[RGB={47,96,165}]{structure}
%\usepackage[activeacute,spanish]{babel}
\usepackage[spanish]{babel}
\usepackage[utf8x]{inputenc}

\setbeamertemplate{blocks}[rounded][shadow=true]

%\usetheme{Singapore}  %Sombreado, denso * ; Warsaw ; Singapore
\usetheme{Warsaw}
\setbeamercovered{transparent}
\usefonttheme{structuresmallcapsserif}


\title{Aprendiendo a escribir documentos con \LaTeX}
\author{Benjam\'in P\'erez}
\institute[SCESI]{Sociedad Cient\'ifica de Estudiantes de Sistemas Inform\'atica \\ $\ $ \\ \includegraphics[height=0.7cm,hiresbb=true]{images/banner}}
\date[\today]{Happy Hacking ;-)}
\logo{\includegraphics[height=1cm,hiresbb=true]{images/scesi02_01.png}}

\begin{document}

\frame{\titlepage}

\section{Introducci\'on}

\begin{frame}{Introduci\'on}
    \framesubtitle{?`Que es \LaTeX?}
    \begin{block}{}
        Primeramente hay que se\~nalar que \LaTeX no es un procesador de textos como:
    \end{block}
    \begin{center}
        \includegraphics[height=5cm,hiresbb=true]{images/procesadores}
    \end{center}
\end{frame}

\begin{frame}{Introducci\'on}
    \framesubtitle{?`Por que \LaTeX?}
    \begin{block}{Razones para usar \LaTeX}
        \begin{itemize}
            \item <1-> Alta calidad tipogr\'afica.
            \item <2-> Documentos bien estructurados.
            \item <3-> Fácil generación de todo tipo de índices, listados de bibliografía citada.
            \item <4-> Excelente calidad en las ecuaciones matemáticas.
            \item <5-> Permite centrarse en el contenido, no en la forma.
        \end{itemize}
    \end{block}
\end{frame}

\begin{frame}{En otras palabras \LaTeX}
  \framesubtitle{Es una Belleza}
  \begin{figure}[t]
  \begin{center}
    \includegraphics [height=8cm, width=11cm] {images/img001.jpg}
  \end{center}
  \end{figure}

\end{frame}

\begin{frame}
  \frametitle{Desventajas}
  \begin{figure}[t]
  \begin{center}
    \includegraphics [height=8cm, width=11cm] {images/img010.png}
  \end{center}
  \end{figure}
\end{frame}

\begin{frame}{Desventajas}
    \begin{itemize}
        \item<1-> Los comandos/macros.
        \item<2-> Los archivos que genera: .aux .log ...
    \end{itemize}
\end{frame}

\section{El c\'odigo \LaTeX}

\begin{frame}[fragile]{El c\'odigo \LaTeX}
    \framesubtitle{Hola mundo!}
    \begin{example}[Ejemplo: Hola Mundo!!]
\begin{verbatim}
% Mi primer documento en LaTeX
\documentclass{article}

\begin{document}
    Hola Mundo!!
\end{document}
\end{verbatim}
    \end{example}
\end{frame}

\begin{frame}[fragile]{El c\'odigo \LaTeX}
    \framesubtitle{Comandos para compilar}
    \begin{itemize}
        \item <1->
        \begin{block}{}
\begin{verbatim}
    $ pdflatex holamundo.tex
\end{verbatim}
        \end{block}
        \item <2->
        \begin{block}{}
\begin{verbatim}
    $ evince holamundo.pdf 
\end{verbatim}
        \end{block}
    \end{itemize}
\end{frame}

\begin{frame}[fragile]{El c\'odigo \LaTeX}
    \framesubtitle{?`Que significa todo esto?}
    \begin{definition}[\% comentarios]
\begin{verbatim}
    % Mi primer documento en \LaTeX
\end{verbatim}
    La primera linea es lo que conocemos como comentarios, el simbolo de \%, cuando \LaTeX $\ $ lo ve ignora 
    inmediatamente el resto de la linea.
    \end{definition}
\end{frame}

\begin{frame}[fragile]{El c\'odigo \LaTeX}
    \framesubtitle{?`Que significa todo esto?}
    \begin{definition}[documentclass\{\}]
\begin{verbatim}
\documentclass{article}
\end{verbatim}
Esta l\'inea es un comando y le dice a \LaTeX que use la clase de documento art\'iculo. Este define que 
tipo de formato queremos usar. Si quieres usar otro tipo de formato solo cambias article por: 
    \end{definition}
    \begin{block}{}
        \begin{itemize}
            \item <1-> article
            \item <2-> letter
            \item <3-> book
            \item <4-> beamer
        \end{itemize}
    \end{block}
\end{frame}

\begin{frame}[fragile]{El c\'odigo \LaTeX}
    \framesubtitle{?`Que significa todo esto?}
    \begin{definition}[begin\{document\}]
\begin{verbatim}
\begin{document}
\end{verbatim}
Esta l\'inea es la que empieza el ambiente llamado document, este comando en \LaTeX $\ $ es el inicio del 
documento.
    \end{definition}
\end{frame}

\begin{frame}[fragile]{El c\'odigo \LaTeX}
    \framesubtitle{?`Que significa todo esto?}
    \begin{definition}[Hola mundo!!]
\begin{verbatim}
Hola mundo!!
\end{verbatim}
Esta l\'inea solo es el texto que hay en tu documento.
    \end{definition}
\end{frame}

\begin{frame}[fragile]{El c\'odigo \LaTeX}
    \framesubtitle{?`Que significa todo esto?}
    \begin{definition}[end\{document\}]
\begin{verbatim}
\end{document}
\end{verbatim}
Esta l\'inea es la que termina el documento.
    \end{definition}
\end{frame}

\begin{frame}[fragile]{documentclass\{article\}}
    \framesubtitle{Algo un poco mas completo}

    \begin{block}{}
\begin{verbatim}
\documentclass[a4paper, 12pt]{article}
\usepackage[spanish]{babel}
\usepackage[utf8x]{inputenc}

\title{Curso de \LaTeX}
\author{Benjamín Pérez}
\date{\today}

\begin{document}

\maketitle

\end{document}
\end{verbatim}
    \end{block}
\end{frame}

\begin{frame}[fragile]{?`Preguntas?}
    \begin{alertblock}{Cosas nuevas}
        \huge {?`Que hay de nuevo?}
    \end{alertblock}
\end{frame}

\begin{frame}[fragile]{Cosas nuevas}
    \begin{exampleblock}{}
    \begin{itemize}
        \item <1->
\begin{verbatim}
\documentclass[a4paper, 12pt]{article}
\end{verbatim}
        \item<2->
\begin{verbatim}
\usepackage[spanish]{babel}
\usepackage[utf8x]{inputenc}
\end{verbatim}
        \item<3->
\begin{verbatim}
\title{Curso de \LaTeX}
\author{Benjamín Pérez}
\date{\today}
\end{verbatim}
        \item<4->
\begin{verbatim}
\maketitle
\end{verbatim}
    \end{itemize}
    \end{exampleblock}
\end{frame}

\begin{frame}[fragile]{Un poco de la Sintaxis}
    \begin{block}{}
    \begin{verbatim}
\usepackage[options]{package}
    \end{verbatim}
    \end{block}
\end{frame}

\begin{frame}[fragile]{Abstract/Resumen}
    \begin{example}
        \begin{verbatim}
\begin{abstract}
    Aqui entra el resumen de tu artículo.
\end{abstract}
        \end{verbatim}
    \end{example}
\end{frame}

\begin{frame}[fragile]{documentclass\{article\}}
    \framesubtitle{Algo un poco mas completo}

    \begin{block}{}
\begin{verbatim}
\documentclass[a4paper, 12pt]{article}
\usepackage[spanish]{babel}
\usepackage[utf8x]{inputenc}

\title{Curso de \LaTeX}
\author{Benjamín Pérez}
\date{\today}

\begin{document}
\maketitle
\begin{abstract}
    Aqui entra el resumen del artículo.
\end{abstract}

\end{document}
\end{verbatim}
    \end{block}
\end{frame}

\begin{frame}[fragile]{section \& subsection}
    \framesubtitle{Estructurar el documento}
    \begin{exampleblock}{}
\begin{verbatim}
\section{Introducci\'on}
    \LaTeX es un sistema de composición de textos, ..
\subsection{?`Porque \LaTeX?}
    Por bla bla bla .....

\section{El c\'odigo \LaTeX}

\subsection{Como compilar}    
\end{verbatim}
    \end{exampleblock}
\end{frame}

\begin{frame}{Herramientas}
    \begin{block}{}
        \begin{itemize}
            \item Kile
            \item Lyx
            \item Kdissert
            \item ShareLaTeX
            \item ...
        \end{itemize}
    \end{block}
\end{frame}

\begin{frame}{?`Preguntas?}
    \large {?`Preguntas?}
\end{frame}

\begin{frame}
    \begin{block}{}
        \centerline {\large {Gracias!! :-)}}
    \end{block}
\end{frame}

\begin{frame}{Algo de Matem\'aticas}
$\int^{x=1}_0 \sum^{x=1}_0$
\end{frame}
\end{document}
